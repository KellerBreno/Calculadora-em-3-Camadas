Este trabalho é um projeto-\/exemplo de construção de uma calculadora com arquitetura cliente/servidor dividida em três camadas, utilizando A\+P\+Is do QT.

\subsection*{Ponto de Partida}

Nessa seção é explicado os requisitos para se compilar o projeto e como executá-\/lo no seu computador.

\subsubsection*{Dependências}

Para compilar este projeto é necessário que você tenha instaladas as A\+P\+Is QT S\+QL, QT Network, QT Database, QT Threads e QT Charts. Essas A\+P\+Is estão disponíveis na ferramenta {\itshape QT Maintenance Tool} instalada junto com o QT.

\subsubsection*{Executando o Projeto}

Após compilar o projeto e gerar os executáveis do cliente e servidor, para executar o código do servidor é necessário\+:

\paragraph*{Windows/\+Linux}

Copiar o arquivo calc\+\_\+example.\+sqlite da pasta Recursos para a localização do executável Calc\+Server.

\paragraph*{Mac\+OS}

Copiar o arquivo calc\+\_\+example.\+sqlite da pasta Recursos para dentro do .app gerado no diretório onde o executável do Calc\+Server está localizado.

\subsubsection*{Banco de Dados vazio}

Caso deseje criar uma nova instância do banco de dados, é possível utilizar o arquivo calc\+\_\+example.\+sql localizado na pasta Recursos para se criar um nova instância do banco por meio do programa {\itshape DB Browser for S\+Q\+Lite} disponível para Linux e Mac\+OS.

\subsection*{Versões}

Após clonar o projeto para a sua máquina, você pode acessar as versões do projeto. Essas versões estão listadas abaixo\+:


\begin{DoxyItemize}
\item \href{https://bitbucket.org/KellerBreno/calculadora/commits/tag/V1}{\tt V1} -\/ Versão 1\+: Projeto com classes concretas;
\item \href{https://bitbucket.org/KellerBreno/calculadora/commits/tag/V2}{\tt V2} -\/ Versão 2\+: Projeto com documentação das classes concretas.
\item \href{https://bitbucket.org/KellerBreno/calculadora/commits/tag/V3}{\tt V3} -\/ Versão 3\+: Projeto com A\+P\+Is separadas.
\item \href{https://bitbucket.org/KellerBreno/calculadora/commits/tag/V3a}{\tt V3a} -\/ Versão 3a\+: Projeto com G\+UI e lógica de negócio separadas.
\item \href{https://bitbucket.org/KellerBreno/calculadora/commits/tag/V4}{\tt V4} -\/ Versão 4\+: Utilizando Actor-\/\+Role.
\end{DoxyItemize}

Para acessar as versões tagueadas é {\bfseries necessário} realizar um checkout no commit referente a aquela tag ou realizar um checkout na tag correspondente.

\subsubsection*{Exemplo}

Caso deseje ver o projeto na Versão 1, você pode fazer\+:

Para checkouts pelo código do commit\+:


\begin{DoxyCode}
1 git checkout a5d02da 
\end{DoxyCode}


Para checkouts pela tag\+: 
\begin{DoxyCode}
1 git checkout V1 
\end{DoxyCode}


\subsection*{Documentação}

Nessa seção é explicado como gerar a documentação do código utilizando a ferramenta {\itshape Doxygen}.

\subsubsection*{Pré-\/\+Requisitos}

Para se utilizar essa ferramenta é necessário instalá-\/la no seu computador. Para isso você pode fazer\+:

\paragraph*{Linux}

Você pode adicionar a ferramenta pelo terminal por meio do apt-\/get\+:


\begin{DoxyCode}
1 sudo apt-get install doxygen 
\end{DoxyCode}


A documentação gerada também utiliza a ferramenta {\itshape dot} oferecida pelo {\itshape graphviz} para gerar gráficos da relação entre as classes. Portanto para a ferramenta funcionar corretamente é necessário instalar o graphviz\+:


\begin{DoxyCode}
1 sudo apt-get install graphviz
\end{DoxyCode}


Por fim, caso deseje utilizar a ferramenta com G\+UI do {\itshape Doxygen} para editar o arquivo de configuração é necessário\+:


\begin{DoxyCode}
1 sudo apt-get install doxygen-gui
\end{DoxyCode}


\subsubsection*{Como utilizar}

Nessa seção será explicado como gerar a documentação através da ferramenta {\itshape Doxygen}.

\paragraph*{Terminal}

Para gerar a documentação sem os testes, basta digitar no terminal (aberto na pasta raiz do projeto) o comando\+:


\begin{DoxyCode}
1 doxygen Doxygen.config 
\end{DoxyCode}


Com isso ele irá gerar uma pasta nomeada \char`\"{}\+Documentação\char`\"{} contendo duas pastas \char`\"{}html\char`\"{}, a qual conterá a documentação no formato de uma página W\+EB e \char`\"{}latex\char`\"{}, a qual conterá a documentação no formato latex que poderá ser utilizado para gerar um arquivo pdf. No caso da documentação gerada por latex é preciso que o compilador para o latex esteja pré-\/configurado na máquina.

\paragraph*{Interface Gráfica}

Caso deseje utilizar a ferramenta gráfica para isso, você deve digitar no terminal\+:


\begin{DoxyCode}
1 doxywizard Doxygen.config
\end{DoxyCode}


Ao executar o comando, ele abrirá uma janela. Nessa janela você pode alterar as configurações ou gerar a documentação. 